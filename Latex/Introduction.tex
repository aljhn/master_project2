\chapter{Introduction} \label{cpt:intro}

%\begin{center}
%\setlength{\fboxsep}{5pt}
%\fbox{%
%\parbox{\textwidth}{With greater focus than ever on zero emissions and lower carbon footprints, more travelers look for environment-friendly means of transportation. Today's railway is characterized by growing maintenance backlogs and struggles to meet expectations regarding punctuality. With strict performance requirements, it is apparent that more effective maintenance strategies are demanded. The current work aims to enable track data recorded in complex terrain to be used as input to data-driven track defect detection models. Made possible, such detection models can form the basis for effective maintenance strategies.}}
%\end{center}

\section{Background and Motivation}\label{sec:background}
%\begin{itemize}
%\item Give concrete background and motivation for why you doing this research. Try to back this with numbers from credible sources. 
%\item Give a quick overview of what others have done so far
%\item Identify knowledge gaps and explicitly mention which of these gaps you want to fill. 
%\end{itemize}

Dynamical systems is an area of mathematics used for describing and modeling anything that involves change. It has become ubiquitously used within the fields of engineering, physics, chemistry, economy and many other applied sciences. Systems of interest can be modeled by sets of equations that describe how the system changes over time. These equations can then be used for investigating desired properties of the system, creating simulations to see how the system behaves and modifying aspects to change the behavior. Using physical laws makes it often possible to write down explicit system models starting from first principles. The model of the system will then be a representation of the underlying dynamics which holds for a given set of assumptions. These assumptions must be made explicitly before the model is to be used, and can lead to simpler or more complicated models. For example will Newton's laws of motion be an accurate model for most practical use cases, but Einstein's theory of relativity proved to be a more general and complicated model which conflicts with Newton's laws in extreme cases.

The field of machine learning has become very popular in recent times because of an increased amount of both data and computational power. Machine learning makes it possible for computers to learn from data in order to solve specific problems. This is often accomplished by making the computer create its own internal model of the problem it is trying to solve. Machine learning has already been used extensively for modeling dynamical systems in various ways, which can be particularly useful for systems where it is difficult to write down explicit equations. Playing board games like Chess, or understanding and generating language are both examples of problems where writing down governing equations are challenging.

An interesting problem is then if it is possible to combine the two approaches, where both explicit equations based on first principles are combined with internally learned representations from data to create models that capture reality in a better and more efficient way than either methods separately.

%\section{Contribution, Research Objectives and Research Questions}\label{sec:objectives_and_contributions}
%\subsection{Contribution}
\section{Contribution}
%Tell briefly which knowledge gaps you intend to fill.

This project explores the potential of using machine learning approaches for learning models of complicated dynamical systems, while also incorporating prior knowledge of the systems derived from first principles. An investigation into what systems are easy to learn and which are more difficult, as well as how to formulate the machine learning problem correctly and efficiently. The main focus is on a deep learning architecture called Physics Informed Neural Networks, which will be explained later in more detail.

This information gained from this initial investigation is then put to use by applying these techniques in clever ways to more complicated problems that might not be easily solved with classical methods. Problems from data-driven discovery of dynamical systems and optimal control of systems governed by partial differential equations are of particular interest.

%\subsection{Objectives}
%To guide this study, a set of research objectives are formulated. In addition, research questions leading to the research objective are presented.



%\underline{\textit{Primary Objective:}}
%Concisely, in an easily understandable form write down the primary objective. This can be more generic

%\textit{Secondary Objectives:}
%The secondary objectives should be precise.
%\begin{itemize}
%    \item SO1
%    \item SO2
%    \item SO3
%\end{itemize}

%\subsection{Research Questions}
\section{Research Questions}

%Based on the objectives frame a set of research questions (preferably three). Each research questions should be connected to the secondary objectives.

A set of research questions are formulated explicitly, to be used for guiding the project in the intended direction.

\begin{itemize}
    \item What is the state of the art of modeling dynamical systems by combining prior information with machine learning?
    \item How are machine learning approaches used for data-driven discovery of dynamical systems when starting from incomplete models?
    \item How can knowledge of dynamical systems be combined with machine learning approaches to find optimal control policies?
\end{itemize}

\section{Structure of the Thesis}\label{sec:structure}
%Give an overview of the structure of thesis. Try to briefly justify the structure. 

\begin{itemize}
    \item Chapter 1 - Introduction: Gives an overview of the main motivations behind the project work, and formulates a set of research objectives.
    \item Chapter 2 - Theory: Presents the background theory necessary to understand the rest of the project work. Also contains an overview of the current literature.
    \item Chapter 3 - Method: Includes an explanation and motivation of each individual experiment, along with the necessary details in order to reproduce the results.
    \item Chapter 4 - Results and Discussion: Shows the results from each experiment presented in the method chapter, and discusses the relevance and meaning of each result.
    \item Chapter 5 - Conclusion and Further Work: Wraps the experimental results back in with the research objectives presented in the introduction, and comes up with further ideas worth investigating.
\end{itemize}



